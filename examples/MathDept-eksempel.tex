\documentclass[norsk]{beamer}


\usepackage[utf8]{inputenx} % For æ, ø, å
\usepackage{babel}          % For oversettelser
\usepackage{amssymb}        % Matematiske symboler
\usepackage{mathtools}      % Matematiske symboler
\usepackage[absolute, overlay]{textpos} % Vilkårlig plassering
\setlength{\TPHorizModule}{\paperwidth} % Textposenheter
\setlength{\TPVertModule}{\paperheight} % Textposenheter
\usepackage{tikz}
\usetikzlibrary{overlay-beamer-styles}  % Overlayeffekter for TikZ


% Fjern logo fra ordinære slides med \usetheme[NoLogo]{MathDept}
\usetheme{MathDept}


\author{Martin Helsø}
\title{Beamereksempel}
\subtitle{Bruk av temaet \texttt{MathDept}}


\begin{document}


\section{Oversikt}


% Bruk
%
%     \begin{frame}[allowframebreaks]
%
% hvis innholdsfortegnelsen ikke får plass på ett lysark.
\begin{frame}
    \frametitle{Innholdsfortegnelse}
    \tableofcontents[currentsection]
\end{frame}


\section{Matematikk}


\subsection{Teorem}


\begin{frame}
    \frametitle{Matematikk}

    \begin{theorem}[Fermats lille sats]
        For et primtall $p$ og $a \in \mathbb{Z}$ er $a^p \equiv a \pmod{p}$.
    \end{theorem}

    \begin{proof}
        De inverterbare elementene i en kropp danner en gruppe under multi\-plikasjon.
        Spesielt danner elementene 
        \[
            1, 2, \ldots, p-1 \in \mathbb{Z}_p
        \]
        en gruppe under multiplikasjon modulo $p$.
        Denne gruppen har orden $p - 1$.
        For $a \in \mathbb{Z}_p$ og $a \neq 0$ har vi dermed at $a^{p-1} = 1 \in \mathbb{Z}_p$.
        Setningen følger umiddelbart.
    \end{proof}
\end{frame}


\subsection{Eksempel}


\begin{frame}
    \frametitle{Matematikk}

    \begin{example}
        Funksjonen $\phi \colon \mathbb{R} \to \mathbb{R}$
        gitt ved $\phi(x) \coloneqq 2x$
        er kontinuerlig i punktet $x \coloneqq \alpha$,
        fordi hvis $\epsilon > 0$ og $x \in \mathbb{R}$
        er slik at $|x - \alpha| < \delta = \frac{\epsilon}{2}$,
        da er
        \begin{equation*}
            |\phi(x) - \phi(\alpha)| = 2|x - \alpha| < 2\delta = \epsilon.
        \end{equation*}
    \end{example}
\end{frame}


\section{Fremheving}


\begin{frame}
    \frametitle{Fremheving}

    Av og til er det nyttig å kunne \alert{utheve} enkelte ord midt i teksten.

    \begin{alertblock}{Viktig melding}
        Hvis man har mye tekst som skal \alert{utheves} kan det være lurt å plassere den i en egen boks.
    \end{alertblock}

    Det er lett å få tekst til å passe \structure{fargetemaet}.
\end{frame}


\section{Lister}


\begin{frame}
    \frametitle{Lister}

    \begin{itemize}
        \item
        Punktlister markeres med en grå boks.
    \end{itemize}

    \begin{enumerate}
        \item
        Nummererte lister markeres med en større grå boks og hvitt nummer.
    \end{enumerate}

    \begin{description}
        \item[Beskrivelser]
        fremhever viktige begreper med grå tekst.
    \end{description}

    \begin{alertblock}{Alertblock}
        \begin{itemize}
            \item
            Lister skifter farge etter omgivelsene.
        \end{itemize}
    \end{alertblock}

    \begin{example}
        \begin{itemize}
            \item
            Lister skifter farge etter omgivelsene.
        \end{itemize}
    \end{example}
\end{frame}


\section{Effekter}


\begin{frame}{Effekter}
    \begin{columns}[onlytextwidth]
        \begin{column}{0.49\textwidth}
            \begin{enumerate}[<+-|alert@+>]
                \item
                Effekter som styrer

                \item
                når tekst vises på skjermen

                \item
                angis med <> og en liste med slides.
            \end{enumerate}

            \begin{theorem}<2>
                Dette teoremet er bare synlig på slide nummer 2.
            \end{theorem}
        \end{column}
        \begin{column}{0.49\textwidth}
            Bruk \textbf<2->{textblock} til vilkårlig plasseringen av objekter.

            \pause
            \medskip

            Det lager en boks
            med angitt bredde (her i prosent av lysarkets bredde)
            og øverste venstre hjørne i angitt koordinat (x, y) (her er x i prosent av bredde og y i prosent av høyde).
        \end{column}
    \end{columns}
    
    \only<1, 3>
    {
        \begin{textblock}{0.3}(0.45, 0.55)
            \includegraphics[width = \textwidth]{MathDept-images/MathDept-apollon}
        \end{textblock}
    }
\end{frame}


\section{Referanser}


\begin{frame}[allowframebreaks]
    \frametitle{Referanser}

    \begin{thebibliography}{}

        % Article er forhåndsvalgt.
        \setbeamertemplate{bibliography item}[book]

        \bibitem{Hartshorne1977}
        R.~Hartshorne.
        \newblock \emph{Algebraic Geometry}.
        \newblock Springer-Verlag, 1977.

        \setbeamertemplate{bibliography item}[article]

        \bibitem{Artin1966}
        M.~Artin.
        \newblock On isolated rational singularities of surfaces.
        \newblock \emph{Amer. J. Math.}, 80(1):129--136, 1966.

       \setbeamertemplate{bibliography item}[online]

       \bibitem{Vakil2006}
       R.~Vakil.
       \newblock \emph{The moduli space of curves and Gromov--Witten theory}, 2006.
       \newblock \url{http://arxiv.org/abs/math/0602347}

       \setbeamertemplate{bibliography item}[triangle]

       \bibitem{AM1969}
       M.~Atiyah og I.~Macdonald.
       \newblock \emph{Introduction to commutative algebra}.
       \newblock Addison-Wesley Publishing Co., Reading, Mass.-London-Don
       Mills, Ont., 1969

       \setbeamertemplate{bibliography item}[text]

       \bibitem{Fraleigh1967}
       J.~Fraleigh.
       \newblock \emph{A first course in abstract algebra}.
       \newblock Addison-Wesley Publishing Co., Reading, Mass.-London-Don Mills, Ont., 1967

       \end{thebibliography}
\end{frame}


\end{document}